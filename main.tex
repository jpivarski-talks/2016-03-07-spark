\documentclass{beamer}
%
% Choose how your presentation looks.
%
% For more themes, color themes and font themes, see:
% http://deic.uab.es/~iblanes/beamer_gallery/index_by_theme.html
%
\mode<presentation>
{
  \usetheme{default}      % or try Darmstadt, Madrid, Warsaw, ...
  \usecolortheme{default} % or try albatross, beaver, crane, ...
  \usefonttheme{default}  % or try serif, structurebold, ...
  \setbeamertemplate{navigation symbols}{}
  \setbeamertemplate{caption}[numbered]
  \setbeamertemplate{footline}[frame number]
} 

\usepackage[english]{babel}
\usepackage[utf8x]{inputenc}

\title[2016-03-07-spark]{ROOT-Spark integration}
\author{Jim Pivarski}
\institute{Princeton Univeristy --- DIANA}
\date{March 7, 2016}

\begin{document}

\begin{frame}
  \titlepage
\end{frame}

% Uncomment these lines for an automatically generated outline.
%\begin{frame}{Outline}
%  \tableofcontents
%\end{frame}

\begin{frame}{Strategy}
\begin{block}{Scope}
Spark is just one big data technology, but almost all of them share a common difficulty: the data backbone is implemented in the Java Virtual Machine (JVM).

\vspace{0.2 cm}
This doesn't mean that users must use Java (I use a mix of Scala and Python), but it does mean that data must be efficiently converted into a JVM-friendly format to be used at a large scale.
\end{block}

\begin{block}{Methods}
I've been keeping three options open:
\begin{itemize}
\item Pure Java using FreeHEP: this works, but I have concerns about its efficiency for non-local files (e.g. xrootd). Will be featured in a performance study when the other methods are ready.

\item 


\end{itemize}
\end{block}

\end{frame}

\end{document}
